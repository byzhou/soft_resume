\documentclass[]{article}

%\usepackage{fancyhdr}
\usepackage{graphicx}
\usepackage{url}
%\pagestyle{fancy}
%\lhead{\footnotesize \parbox{11cm}{Mahmoud Zangeneh} }

\parindent=0pt
\usepackage[margin=0.5in]{geometry}

\begin{document}
\pagestyle{empty}
\begin{center}
{\large\textbf{Boyou Zhou}}\\
Department of Electrical and Computer Engineering, Boston University\\
8 St Mary's St, Photonics Center, Room 340, Boston, MA 02215\\
Email: bobzhou@bu.edu, Phone: 617-678-8480\\
\rule[-0.1cm]{7.5in}{0.01cm}
\end{center}
%

\textbf{Education}
\begin{table*}[h]
  %\newcommand{\z}{$^*$}
  %\centering
  \begin{tabular}{p{2.0in}p{2.0in}r}
    Boston University& Department of ECE & Ph.D. candidate, 2013 - present
    (GPA: 3.8/4.0)\\ 
    Southeast University & Department of ECE & B.S., 2013
    (Major GPA: 3.86/4.0)\\ 
  \end{tabular}
  \label{tbl:1}
\end{table*}

\noindent \textbf{Internship Experience}
    \begin{itemize}
        \item \textbf{Design Engineer at Analog Device Inc. San Jose, CA, June.2014-August.2014}\\
        Real Number Modeling ADP1048 PFC chip's application power stage with
        voltage-input-current-output feedback loop and chip's analog components, using Cadence
        Verilog-ams and verilog testbench for close loop simulation. Exploring the model with System
        Verilog to achieve consistency between \textit{Analog Device Inc.} Adice5 analog simulation
        and \textit{Cadence Incisive} digital simualtion. 
        \item \textbf{Design Engineer at Analog Device Inc. Wilmington, MA, June.2015-August.2015}\\
        Research on exploring Real Number Modeling using System Verilog. Set up environment for
        System Verilog Real Number Modeling, including Netlisting tool integration, netlisted code
        post-processing and data type library customization for RNM.\@ Got job
        offered from ADI in the end of this internship.
		\item \textbf{Design Engineer at Qualcomm Inc. Santa Clara, CA, September.2016-December.2016}\\
		Implementing and modeling DSP modules in touch screen chips. Set up
		automated verilog connector for high level module connection.
    \end{itemize}

\noindent \textbf{Publication}
\begin{itemize}
    \item \textbf{Detecting hardware Trojans using backside optical imaging of
    embedded watermarks, \textit{2015 Design Automation Conference}}, first author
    
    Engineering a specific standard cell for photonic response Hardware Trojan Detection, comparing
    process variation using Monte Carlo simulation results with the timing and power analysis
    results from \textit{Cadence Encounter} ASIC Synthesis tool.

    \item \textbf{Rapid mapping of digital integrated circuit logic gates via multi-spectral backside
    imaging, \textit{2015 Optical Express}}

    Near-IR high resolution imaging and classifications for digital integrated circuits to detect
    Hardware Trojans.

	\item \textbf{Rapid mapping of digital integrated circuit logic gates via
multi-spectral backside imaging, \textit{arXiv 2016}}

	Gate classifications using near-IR optical imaging for HT detections in ICs.

\end{itemize}

\noindent \textbf{Patent}
\begin{itemize}
    \item \textbf{Fill Cell Watermark for Circuit Layout Validation}, patented application submitted
\end{itemize}

\noindent \textbf{Relevant Courses}
\begin{itemize}
\item Graduate:
                    Hardware Security, Linux Kernel Programing in Embedded System, RF/IC Design,
                    Advanced Digital Design with Verilog and FPGA, Computer
                    Architecture, Digital VLSI Design, RF Analog VLSI Design,
                    Operating System, Hardware Security, Data Structure.
\item Undergraduate:
                    Digital Signal Processing, Communication System, Signals
                    and Systems, Computer Organization and 
                    Architecture, Electromagnetic Wave and Field. 
\end{itemize}

%CLASS PROJECTS AND UNDERGRADUATE PROJECTS
\noindent \textbf{Ongoing Projects}
\begin{itemize}

		\item \verb+Encryption Implementation+

		Implemented hardware encryption on Zedboard platform to achieve
hardware acceleration and reconfiguration of encryption algorithms.
Implemented software encryption interface on the Linux side to the Zynq FPGA
hardware acceleration.

		\item \verb+Program Profiling using Hardware Performance Counters and Machine Learning+
		
		Used Hardware Performance Counters (HPCs) informations to train various
machine learning models to classify between various programs. Implemented the HPC
measurements by clustering machines based on Rabbitmq messaging system, samba
server and pixie booting service.

\end{itemize}

% \noindent \textbf{Skills}
% \begin{itemize}
% \item Programming Languages:  Verilog/Verilog-ams/SystemVerilog, Perl, Python, C, C++, Ruby, Bash,
% Hspice, Java, markdown, latex
% \item Softwares: git/git-svn, vim, gcc/g++, qemu, gdb, emacs, Matlab, Makefile, Lingo,
% pdflatex/latexmk, eclipse
% \item Circuit Design Tools:  incisive, simvision, Adice5, Modelsim, Altium Designer, Cadence
% rc$\&$encounter, HSPICE, Cadence Virtuoso, Virtuoso Layout Editor, Xilinx ISE, Quartus II, HFSS,
% Microwave Office. 
% \item Hardware Platform: Zedboard, Xilinx Spatan 6, Gumstix
% 
% \end{itemize}
% %\newpage
% \noindent \textbf{Teaching Experience}
% \begin{itemize}
% \item Graduate Teaching Fellow for Operating System (Raspberry pi working
% platform) at Boston University (Spring 2014).
% \item Graduate Teaching Fellow for Introduction to Software Engineering (C++)
% (Fall 2013).
% \end{itemize}

\end{document}
