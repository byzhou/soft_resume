\documentclass[]{article}

%\usepackage{fancyhdr}
\usepackage{graphicx}
\usepackage{url}
%\pagestyle{fancy}
%\lhead{\footnotesize \parbox{11cm}{Mahmoud Zangeneh} }

\parindent=0pt
\usepackage[margin=0.5in]{geometry}

\begin{document}
\pagestyle{empty}
\begin{center}
{\large\textbf{Boyou Zhou}}\\
Department of Electrical and Computer Engineering, Boston University\\
8 St Mary's St, Photonics Center, Room 340, Boston, MA 02215\\
Email: bobzhou@bu.edu, Phone: 617-678-8480, cv.boyouz.com\\
\rule[-0.1cm]{7.5in}{0.01cm}
\end{center}
%

\textbf{Education}
\begin{table*}[h]
  %\newcommand{\z}{$^*$}
  %\centering
  \begin{tabular}{p{2.0in}p{2.0in}r}
    Boston University& Department of ECE & Ph.D. candidate, 2013 - present
    (GPA: 3.8/4.0)\\ 
    Southeast University & Department of ECE & B.S., 2013
    (Major GPA: 3.86/4.0)\\ 
  \end{tabular}
  \label{tbl:1}
\end{table*}

\noindent \textbf{Ongoing Projects}
\begin{itemize}

		\item \verb+Malware Detection using Hardware Performance Counters and Machine Learning+

In the state-of-art malware detection, the research community has developed the
malware detection using machine learning to training profiles of Hardware
Performance Counters. In order to explore the feasibility of this method,  I
have built a cluster of experimental machines and connected them with Rabbitmq
message system. I evaluated the results of Hardware Performance Counters using
popular machine learning algorithms, such as K-Nearest Neighbors, Multi-layer
Perceptions, and etc.

		\item \verb+Evercookie Tracker+

In the web-based communications, web developers have leveraged cookies for
storing identities, saving personal data and behavior tracking. However, some
web maliciously tracks user behaviors even though users have already removed
the cookies. I have developed a evaluation backend to crawl newly visited
website to detect the existence of evercookies. The newly detected evercookie
websites are sent to the browser extension to notify the users.

		\item \verb+Hardware Detection using Optical Imaging Methods+

We developed an optical based Hardware Trojan Detection. Hardware Trojans are the malicious
hardware units inserted by intruders. The detection of Hardware Trojans are extremely hard -
the reverse engineering of hardware units is costly and destructive. We use near-Infrared light 
to image the backside of the Integrated Circuit Designs, since the backside is transparent. 
However, due to the long wavelength of near-Infrared light, the imaged results requires additional
data analysis. Here, I have used the Support Vector Machine and K Nearest Neighbors to classify 
various imaged results.
		
\end{itemize}


\noindent \textbf{Internship Experience}
    \begin{itemize}
		\item \textbf{Design Engineer at Qualcomm Inc. Santa Clara, CA, September.2016-December.2016}\\
I have implemented a Digital Signal Processing Module in the chip design. I modeled my signal processing
in Python withe builder design pattern. My model brought the flexibility into parameter changing and 
future developments. In addition to designing the models, I also provided the websites for documenting the
tools during the developments. Meanwhile, I designed a tool for Verilog automatic connection using Python 
and other scripting languages.
        \item \textbf{Design Engineer at Analog Device Inc. Wilmington, MA, June.2015-August.2015}\\
        and \textbf{Design Engineer at Analog Device Inc. San Jose, CA, June.2014-August.2014}\\
In these two internships, I mainly explore the Real Number Modeling in chip modeling and also board 
module modeling. I used Verilog, Verilog-ams, and SystemVerilog to module the current and voltage 
behaviors. 
    \end{itemize}

% 
\noindent \textbf{Publication}
\begin{itemize}
\item\url{https://scholar.google.com/citations?user=nj1je50AAAAJ&hl=en}
\end{itemize}
% \begin{itemize}
%     \item \textbf{Detecting hardware Trojans using backside optical imaging of
% 	embedded watermarks}, \textbf{Zhou, B.}, Adato, R., Zangeneh, M., Yang, T., Uyar, A.,
% Goldberg, B., Unlu, S. and Joshi, A., 2015, June. In Design Automation
% Conference (DAC), 2015 52nd ACM/EDAC/IEEE (pp. 1-6). IEEE.
%     
% 
% 	\item \textbf{Integrated nanoantenna labels for rapid security testing of
% semiconductor circuits} Adato, R., Uyar, A., Zangeneh, M., \textbf{Zhou, B.}, Joshi, A.,
% Goldberg, B. and Unlu, S., 2015, October.  In Frontiers in Optics (pp.
% FTh1B-2). Optical Society of America.
% 
% 
% 	\item \textbf{Rapid mapping of digital integrated circuit logic gates via
% multi-spectral backside imaging}, Adato, R., Uyar, A., Zangeneh, M., \textbf{Zhou, B.},
% Joshi, A., Goldberg, B. and Unlu, M.S., 2016. arXiv preprint arXiv:1605.09306.
% 
% 	\item \textbf{High-performance low-energy implementation of cryptographic
% algorithms on a programmable SoC for IoT devices}, \textbf{Zhou, B.}, Egele, M. and
% Joshi, A., 2017, September.  In High Performance Extreme Computing Conference
% (HPEC), 2017 IEEE (pp. 1-6). IEEE.
% \end{itemize}

\noindent \textbf{Relevant Courses}
\begin{itemize}
\item 
Linux Kernel Programing in Embedded System, Computer Architecture, Cyber
Security, Operating System, Hardware Security, Data Structure.
\end{itemize}

%CLASS PROJECTS AND UNDERGRADUATE PROJECTS

\noindent \textbf{Skills}
\begin{itemize}
\item Softwares: Linux-utils, GDB, various IDEs, and basic network tools
\item Programming Languages:  Python, C/C++, Bash and other various scripting languages,
Android and Website Basics
\end{itemize}
%\newpage
\noindent \textbf{Teaching Experience}
\begin{itemize}
\item Graduate Teaching Fellow for Operating System (Raspberry pi working
platform) at Boston University (Spring 2014).
\item Graduate Teaching Fellow for Introduction to Software Engineering (C++)
(Fall 2013).
\end{itemize}

\end{document}
